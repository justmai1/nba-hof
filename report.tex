% Options for packages loaded elsewhere
\PassOptionsToPackage{unicode}{hyperref}
\PassOptionsToPackage{hyphens}{url}
%
\documentclass[
  11pt,
]{article}
\usepackage{amsmath,amssymb}
\usepackage{iftex}
\ifPDFTeX
  \usepackage[T1]{fontenc}
  \usepackage[utf8]{inputenc}
  \usepackage{textcomp} % provide euro and other symbols
\else % if luatex or xetex
  \usepackage{unicode-math} % this also loads fontspec
  \defaultfontfeatures{Scale=MatchLowercase}
  \defaultfontfeatures[\rmfamily]{Ligatures=TeX,Scale=1}
\fi
\usepackage{lmodern}
\ifPDFTeX\else
  % xetex/luatex font selection
    \setmainfont[]{Times New Roman}
\fi
% Use upquote if available, for straight quotes in verbatim environments
\IfFileExists{upquote.sty}{\usepackage{upquote}}{}
\IfFileExists{microtype.sty}{% use microtype if available
  \usepackage[]{microtype}
  \UseMicrotypeSet[protrusion]{basicmath} % disable protrusion for tt fonts
}{}
\makeatletter
\@ifundefined{KOMAClassName}{% if non-KOMA class
  \IfFileExists{parskip.sty}{%
    \usepackage{parskip}
  }{% else
    \setlength{\parindent}{0pt}
    \setlength{\parskip}{6pt plus 2pt minus 1pt}}
}{% if KOMA class
  \KOMAoptions{parskip=half}}
\makeatother
\usepackage{xcolor}
\usepackage[margin=1in]{geometry}
\usepackage{color}
\usepackage{fancyvrb}
\newcommand{\VerbBar}{|}
\newcommand{\VERB}{\Verb[commandchars=\\\{\}]}
\DefineVerbatimEnvironment{Highlighting}{Verbatim}{commandchars=\\\{\}}
% Add ',fontsize=\small' for more characters per line
\usepackage{framed}
\definecolor{shadecolor}{RGB}{248,248,248}
\newenvironment{Shaded}{\begin{snugshade}}{\end{snugshade}}
\newcommand{\AlertTok}[1]{\textcolor[rgb]{0.94,0.16,0.16}{#1}}
\newcommand{\AnnotationTok}[1]{\textcolor[rgb]{0.56,0.35,0.01}{\textbf{\textit{#1}}}}
\newcommand{\AttributeTok}[1]{\textcolor[rgb]{0.13,0.29,0.53}{#1}}
\newcommand{\BaseNTok}[1]{\textcolor[rgb]{0.00,0.00,0.81}{#1}}
\newcommand{\BuiltInTok}[1]{#1}
\newcommand{\CharTok}[1]{\textcolor[rgb]{0.31,0.60,0.02}{#1}}
\newcommand{\CommentTok}[1]{\textcolor[rgb]{0.56,0.35,0.01}{\textit{#1}}}
\newcommand{\CommentVarTok}[1]{\textcolor[rgb]{0.56,0.35,0.01}{\textbf{\textit{#1}}}}
\newcommand{\ConstantTok}[1]{\textcolor[rgb]{0.56,0.35,0.01}{#1}}
\newcommand{\ControlFlowTok}[1]{\textcolor[rgb]{0.13,0.29,0.53}{\textbf{#1}}}
\newcommand{\DataTypeTok}[1]{\textcolor[rgb]{0.13,0.29,0.53}{#1}}
\newcommand{\DecValTok}[1]{\textcolor[rgb]{0.00,0.00,0.81}{#1}}
\newcommand{\DocumentationTok}[1]{\textcolor[rgb]{0.56,0.35,0.01}{\textbf{\textit{#1}}}}
\newcommand{\ErrorTok}[1]{\textcolor[rgb]{0.64,0.00,0.00}{\textbf{#1}}}
\newcommand{\ExtensionTok}[1]{#1}
\newcommand{\FloatTok}[1]{\textcolor[rgb]{0.00,0.00,0.81}{#1}}
\newcommand{\FunctionTok}[1]{\textcolor[rgb]{0.13,0.29,0.53}{\textbf{#1}}}
\newcommand{\ImportTok}[1]{#1}
\newcommand{\InformationTok}[1]{\textcolor[rgb]{0.56,0.35,0.01}{\textbf{\textit{#1}}}}
\newcommand{\KeywordTok}[1]{\textcolor[rgb]{0.13,0.29,0.53}{\textbf{#1}}}
\newcommand{\NormalTok}[1]{#1}
\newcommand{\OperatorTok}[1]{\textcolor[rgb]{0.81,0.36,0.00}{\textbf{#1}}}
\newcommand{\OtherTok}[1]{\textcolor[rgb]{0.56,0.35,0.01}{#1}}
\newcommand{\PreprocessorTok}[1]{\textcolor[rgb]{0.56,0.35,0.01}{\textit{#1}}}
\newcommand{\RegionMarkerTok}[1]{#1}
\newcommand{\SpecialCharTok}[1]{\textcolor[rgb]{0.81,0.36,0.00}{\textbf{#1}}}
\newcommand{\SpecialStringTok}[1]{\textcolor[rgb]{0.31,0.60,0.02}{#1}}
\newcommand{\StringTok}[1]{\textcolor[rgb]{0.31,0.60,0.02}{#1}}
\newcommand{\VariableTok}[1]{\textcolor[rgb]{0.00,0.00,0.00}{#1}}
\newcommand{\VerbatimStringTok}[1]{\textcolor[rgb]{0.31,0.60,0.02}{#1}}
\newcommand{\WarningTok}[1]{\textcolor[rgb]{0.56,0.35,0.01}{\textbf{\textit{#1}}}}
\usepackage{graphicx}
\makeatletter
\def\maxwidth{\ifdim\Gin@nat@width>\linewidth\linewidth\else\Gin@nat@width\fi}
\def\maxheight{\ifdim\Gin@nat@height>\textheight\textheight\else\Gin@nat@height\fi}
\makeatother
% Scale images if necessary, so that they will not overflow the page
% margins by default, and it is still possible to overwrite the defaults
% using explicit options in \includegraphics[width, height, ...]{}
\setkeys{Gin}{width=\maxwidth,height=\maxheight,keepaspectratio}
% Set default figure placement to htbp
\makeatletter
\def\fps@figure{htbp}
\makeatother
\setlength{\emergencystretch}{3em} % prevent overfull lines
\providecommand{\tightlist}{%
  \setlength{\itemsep}{0pt}\setlength{\parskip}{0pt}}
\setcounter{secnumdepth}{5}
\ifLuaTeX
  \usepackage{selnolig}  % disable illegal ligatures
\fi
\usepackage{bookmark}
\IfFileExists{xurl.sty}{\usepackage{xurl}}{} % add URL line breaks if available
\urlstyle{same}
\hypersetup{
  pdftitle={NBA Hall of Fame Prediction},
  pdfauthor={Justin Mai},
  hidelinks,
  pdfcreator={LaTeX via pandoc}}

\title{NBA Hall of Fame Prediction}
\author{Justin Mai}
\date{2025-05-15}

\begin{document}
\maketitle

{
\setcounter{tocdepth}{2}
\tableofcontents
}
\newpage

\section{Abstract}\label{abstract}

\newpage

\section{Introduction}\label{introduction}

The NBA Hall of Fame inducts the most influential players, coaches,
teams, and referees yearly. There have only been just over 150 NBA
players inducted to the Hall of Fame which started with the inaugural
class of 1959. So what classifies an NBA player as a Hall of Famer
compared to other NBA players? This research paper will identify the
probability of current NBA players one day making the Hall of Fame based
on historical trends.

\subsection{Accolades and Awards}\label{accolades-and-awards}

A common debate within the basketball community is the infamous ``LeBron
vs.~Michael Jordan'' debate to crown on player as the Greatest of All
Time (G.O.A.T.). Analysts will often start with the quantitative in game
statistics by looking at all time averages for both players. Looking at
the primary statistics, throughout Jordan's career, he averaged \(30.1\)
points per game, \(6.2\) rebounds per game, and \(5.3\) assists per
game. On the other end, LeBron averages \(27.0\) points per game,
\(7.5\) rebounds per game, and \(7.4\) assists per game. While Jordan
has the edge on scoring, LeBron has the edge in the other primary
statistics so its difficult to make a clear conclusion based on this.
However, this isn't the primary argument for both players, if you've
ever been part of this debate you'll often hear the notion that ``Jordan
went 6 for 6 in championship games''. The accolades and awards that each
player compiles is often the primary argument.

While there is no clear calculator for identifying if a player will make
the Hall of Fame, in all cases, accolades and awards will be a
significant predictor to identifying HOFers. These accolades will
consist of \textbf{Regular Season MVPs, Championship Wins, Finals MVPs,
All-NBA Selections, All-Star Selections, End-of-Season Awards} and
possibly much more. These awards all signify the impact that a player
has had on their respective teams, demonstrating how their contribution
leads to the team's success.

\subsection{Player Impact / Other
Considerations}\label{player-impact-other-considerations}

Within the G.O.A.T debate, a primary argument for LeBron would be his
longevity and consistent impact on the game and the teams he goes to,
the qualitative factors that goes beyond the box score and award counts.
While accolades and awards provide a quantitative summary of a player's
career, qualitative aspects such as leadership, clutch performances,
career longevity, influence on team culture, and global popularity often
shape the broader legacy of a player.

For example, LeBron's ability to lead multiple franchises to the NBA
Finals---winning championships with three different teams---is a
testament to his versatility and value as a player. Similarly, players
like Allen Iverson and Vince Carter are celebrated not only for their
statistics and accolades, but also for their cultural impact, influence
on future generations, and overall contribution to the evolution of the
game.

When making predictions for Hall of Fame inductees, there are also many
traits outside of the box score that contributes to the selection
process. However, this type of impact often correlates with the
accolades and overall quantitative statistics so it will be contributing
factor within the analysis. Something like longevity will also be taken
into account through the number of years they played all together and
the number of years they played for one team.

\section{Methods}\label{methods}

\begin{Shaded}
\begin{Highlighting}[]
\NormalTok{df }\OtherTok{\textless{}{-}} \FunctionTok{read.csv}\NormalTok{(}\StringTok{"data/final.csv"}\NormalTok{)}
\end{Highlighting}
\end{Shaded}

\section{Results}\label{results}

\section{Discussion}\label{discussion}

\subsection{Limitations}\label{limitations}

\subsection{Next Steps}\label{next-steps}

\section{Bibliography}\label{bibliography}

\url{https://www.sportingnews.com/us/nba/news/michael-jordan-vs-lebron-james-goat-debate/sl8xdozy5u1m1s4t5m3npeqo1}

\url{https://www.kaggle.com/datasets/sumitrodatta/nba-aba-baa-stats?select=Player+Career+Info.csv}

\end{document}
